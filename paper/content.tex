% status: 100
% chapter: TBD

\def\paperstatus{50} % a number from 0-100 indicating your status. 100
                % means completed
\def\paperchapter{Impala and Kudu} % This section is typically a single keyword. from
                   % a small list. Consult with theinstructors about
                   % yours. They typically fill it out once your first
                   % text has been reviewed.
\def\hid{hid-sp18-407} % all hids of the authors of this
                                % paper. The paper must only be in one
                                % authors directory and all other
                                % authors contribute to it in that
                                % directory. That authors hid must be
                                % listed first
\def\volume{9} % the volume of the proceedings in which this paper is to
           % be included

\def\locator{\hid, Volume: \volume, Chapter: \paperchapter, Status: \paperstatus. \newline}

\title{Processing Massive Datasets: Apache Kudu and Cloudera Impala}


\author{Keith Hickman}
\affiliation{%
  \institution{Indiana University School of Informatics and Computing}
  \streetaddress{919 E 10th St}
  \city{Bloomington} 
  \state{Indiana} 
  \postcode{47408}
}
\email{keonhick@iu.edu}

% The default list of authors is too long for headers}
\renewcommand{\shortauthors}{G. v. Laszewski}


\begin{abstract}
This paper examines the benefits of using Apache Kudu and Apache Impala. Apache Kudu is an open-source storage engine for structured streaming data that provides record-level update capabilities.  Apache Impala is a high-performance SQL query engine tightly integrated with Kudu among other data storage platforms.  Kudu and Impala together provide many advantages over traditional Hadoop storage and query systems.  
\end{abstract}

\keywords{\locator\ apache, kudu, impala, SQL, analytics}

\maketitle


\section{Introduction}
The evolution of cloud computing technologies changes at an ever-increase pace. Applications of these technologies vary greatly but have one thing in common: the need to ingest, store, process, manipulate, and extract data to make decisions with. The volume of these available data has increased just as quickly, as organizations increasingly view data as a commodity.  One effect of this trend is the shift of focus away from some well-established technologies toward new methods aimed at scalable processing power. 

\paragraph{Background}
SQL was the dominiant language for working with data until the
mid-2000's.  With the explosion in the volume and velocity of
available data, programmers struggled to scale SQL-based storage and
query engines to meet demand; thus NoSQL was born. Now, Relational
 Database Management Systems (RDBMS) platforms are being designed once 
again as processing power overcomes the hurdles for which NoSQL was 
originally designed. Many of the reasons that contributed to the 
decline of RDBMS became unpopular and are no longer relevant, or the 
problems have been solved. Developers can now choose a programming 
paradigm depending on their specific needs without sacrificing 
performance or usability, and without having to create complex 
architectures.  Kudu represents one option in this domain. In 2015, 
Cloudera donated both Impala and Kudu projects to the Apache Software 
Foundation in hopes that a full-time developer community would spark 
the projects into stable, widely adopted platforms. 

\section{Kudu and Impala}
Kudu was started in 2014 and is now a Top-Level Project within Apache. Driven by need to ingest, update, and analyze massive datasets at scale, given a low-cost, open-source framework, Kudu is described as an open-source storage engine for structured data designed to provide low-latency random access to distributed data sources, particularly within the Hadoop ecosystem. Cite 17. ``At a high level, Kudu is a new storage manager that enables durable single-record inserts, updates, and deletes, as well as fast and efficient columnar scans due to its in-memory row format and on-disk columnar format.''  Cite 14. One of the main advantages Kudu offers is a simple API for row-level writes, updates, upserts, and deletes, while maintaining query throughput speeds similar to binary stores such as Apache Avro. Cite 1. Impala is a SQL engine providing low-latency, high-concurrency access for business intelligence-type analytic queries on HDFS.  Cite 2. 

\paragraph{Hadoop Ecosystem}
Kudu and Impala together provide many advantages.  With Kudu, a user can access many data stores through a single query engine like Impala. With Impala, many sources of data can be queried from a single SQL interface, such as Hive, Hbase, or naturally Kudu. Kudu is tightly integrated with Impala, but supports the entire Apache ecosystem including Spark, Hive, Hbase, and others. Cite 14. Because both platforms are part of the Hadoop ecosystem, their capabilities are greatly expanded through interoperability with the data stores, cluster managers, and visualization tools. 


\section{Kudu}
Kudu benefits from several design choices aimed at bridging the gap between static and dynamic data stores. Static Hadoop data stores such as Apache Avro or Parquet have limited ability to write but low-latency interactivity. These ``traditional'' data management stores in represent immutable databases and are more suited to unstructured or semi-structured streaming data. Cite 1. Whereas semi or unstructured stores such as Hive can read and write real-time data but provide less-than-optimal analytical performance. Cite 2. Mutable databases like Hbase allow for low-latency record-level reads and writes, but lag behind the static file formats for sequential read-through for applications like SQL-based analytics and machine learning.  Cite 14. ``Kudu fills the gap between immutability in static data stores and latency during sequential read-through in mutable databases, and offers several improvements and design choices in read, write, distribution, and maintenance operations.''  Cite 1.

\paragraph{Kudu vs. Lambda}
This architecture makes Kudu very attractive for storing streaming data that may need to be modified at a later time, such as healthcare or weather data. Cite 7. ``Today, many users try to solve this challenge via a Lambda architecture, which presents inherent challenges by requiring different code bases and storage for the necessary batch and real-time components.'' Cite 14. Kudu and Impala together circumvents the need for such complex architecures, as they are well-suited for real-time interactivity and are designed to be tightly integrated.  

\section{Flexible Horizontal Partionioning Schemes}
Kudu utilizes a flexible array of horizontal partitioning schemes to achieve high-throughput storage and low-latency random access. Cite 1. Contrast other distributed data stores such as Hbase or Cassandra which use either key-range horizontal partitioning, or hash partition methods respectively. In Kudu, "the partition schema acts as a function which can map from a primary key tuple into a binary partition key.  Each tablet covers a contiguous  range  of  these partition keys. Thus,  a client, when performing a read or write, can easily determine which tablet should hold the given key and route the request accordingly." Cite 1. By optimizing the scanning features and compression codecs, Kudu reduces the amount of time needed to return row-level data. Additionally, the flexibility of choosing the appropriate partitioning scheme makes Kudu an viable option for high-throughput scenarios. 

\paragraph{Raft}
Kudu uses the Raft consensus and horizontal partitioning of tables into ``tablets'' or the subset of a partitioned table. Conversely, Parquet, a static binary data store, uses vertical partitioning, which can provide performance advantages in partial queries and retrievals. 

Raft implements consensus by first electing a distinguished leader, then giving the leader complete responsibility for managing the replicated log. The leader accepts log entries from clients, replicates them on other servers, and tells servers when it is safe to apply log entries to their state machines. Having a leader simplifies the management of the replicated log. For example, the leader can decide where to place new entries in the log without consulting other servers, and data flows in a simple fashion from the leader to other servers. A leader can fail or become disconnected from the other servers, in which case a new leader is elected.  Cite 3. Note, the concept of a leader/follower is different than a master/client relationship often found in 

This sample has some advanced features that may not be offered by
other \LaTeX\ frameworks. All of these features are accessible at this
time throug a makefile. These advanced features are optional and
introduce a number of automated tests for the document. Such tests are
executed through a Makefile. Such makefiles can be executed on Linux
or OSX. They are not available on sharelatex. As we have taught you
how to set up a Linux system in class it is part of the leraning
experience that you use that to verify the correctness of your
paper. Alternatively you can set up a docker container and use
that. It will be your responsibility to create such a container or set
up the Linux environment with a \textit{full} version of \LaTeX\ and biber
installed.

The document is organized in a a repository, that needs to be
replicated into your github repository. {\bf Please make sure that all
  filenames and directory names but a README.md are all lower case and
  have no underscore in it. Please make sure they do not contain a
  space in the filename.} We repeat again: Be mindful about the
capitalization of filenames and directory names. We only use lower
case for any file in our directory and we never use spaces or non ASCI
characters. 

IN order for you to most easily compile the paper you will also need
to check out the directory hid-sample that includes the formats. 

Your recommended directories should look like

\begin{verbatim}
~/github/cloudmesh-community/hid-samle
~/github/cloudmesh-community/hid-sp18-000
\end{verbatim}

where 000 the number part of your hid. The two hid directories are
cloned from github. We taught you how to do this and it is a class
goal that you know how to do it. 

In your hid folder you will find, once it is set up the following
directory and files:


We have the following files that you will modify:

\begin{verbatim}
paper/content.tex
paper/report.tex
paper/report.bib
images/README.md
\end{verbatim}

We separated the content from the report format into two different
files, as we found that students in the past manipulated our format to
circumvent our page limitation and artificially increased spacing and
other unnecessary things that we would consider as cheating.
If you

Thus the only file you are allowed to modify is the content.tex file
and the report.bib bibliography file and the images directory in which
you place your images.

Within the text, images and tables are {\bf must} be placed in floats
as discussed in the handbook. You can place them with \verb|[htb]| at
any place in the document. However, we will for the submission force
the placement at the end of the document. You do not have to place
them at the end. Please keep in mind that tables and figurs and
programs that are also to be placed in figures do not count towards
the paper length. Please also be reminded this is not a presentation
and that lengthy enumerations will lead to point deductions. IN case
of questions, consult with the TAs.

\section{Introduction}

Put here an introduction about your topic. 
We just need one sample reference so the paper compiles in \LaTeX\ so we
put it here~\cite{editor00}.

\section{Figures}

In Figure~\ref{f:fly} we show a fly. Please note that because we use
just columwidth that the size of the figure will change to the
columnwidth of the paper once we change the layout to final. Changing
the layout to final should not be done by you. All figures will be
listed at the end.  Please do not use phrases such as \textit{shown in
  the figure below}. Instead use such as shown in Figure~\ref{f:fly}.

\begin{figure}[!ht]
  \centering\includegraphics[width=\columnwidth]{../../hid-sample/tex/images/fly.pdf}
  \caption{Example caption}\label{f:fly}
\end{figure}

To avoid copying the example figure in each hid directory, we actually
import it from the hid-sample directory. Lets assume you have a figure
in the image directory called myfigure.pdf. You can than replace the
line in the example with

\begin{verbatim}
\centering\includegraphics[width=\columnwidth]{images/myfigure.pdf}
\end{verbatim}

In case you use png than you must have a png that is at least 300dpi
(find out what that means and how to do that) and use 

\begin{verbatim}
\centering\includegraphics[width=\columnwidth]{images/myotehrfigure.png}
\end{verbatim}

When modifying the example, please do not check in the images from the
examples into your images directory as you will not need them for your
paper. Instead use images that you like to include. If you do not have
any images, do not palce anything in the images folder. However most
technologies could benefit for one image. Make sure you do not
plagiarize the image. Find out from the hand book how to do that. Any
plagiarism of images will result that we return the paper without
review as you have not understood how plagiarism works.

\section{Tables}

In case you need to create tables, you can do this with online tools
(if you do not mind sharing your data) such as
\url{https://www.tablesgenerator.com/} or other such tools (please
google for them). They even allow you to manage tables as CSV.

or generate them by hand while using the provided template in Table
\ref{t:mytable}. Note that
the caption is before the tabular environment.

\begin{table}[htb]
\centering
\caption{My caption}
\label{t:mytabble}
\begin{tabular}{lll}
1 & 2 & 3 \\
\toprule
4 & 5 & 6 \\
7 & 8 & 9
\end{tabular}
\end{table}

\section{Quotes}

Do not use double quotes 

\begin{verbatim}
"this is a wrong example"~\cite{las2018paper} 
\end{verbatim}

but use \LaTeX\ quotes

\begin{verbatim}
``This is a right example''~\cite{las2018paper}
\end{verbatim}


Quotes {\bf MUST} not be used to highlight works. Quotes are {\bf
  STRICTLY} used for quoting text from sources with citation
following. If we find a quote that is not followed by a citation we
will return the paper without review. Please note that the two single
quotes at the begining are not the same direction as the single quotes
at the end. On the US keyboard the beginning quotes are next to the
keyborad key 1 and the end quotes are next to the ENTER or RETURN
key. We use them to identify how much you have quoted. If you have
quoted more than 30\% we will not review your work. Most conferences
and journals do not allow quoting above 20\%, so consider yourself
lucky to have such a relaxed standard. 

\section{Emphasize}

To emphasize a text you do not use double or single quotes you use
\emph{emph}.

See the Handbook for more details.


\section{Labels}

Do not use actual numbers in the taxt after you write for example
Figure 1 use the ref for the figure while using its label. In our
example it is Figure~\ref{f:fly} and Table Figure~\ref{t:mytabble}.
See the source for the example.

\section{Footnotes}

Footnotes must be avoided in papers. All URLs must be included as full
references and citations and used with the \verb|\cite| command
\footnote{do not use footnotes}. You {\bf must} not use urls in the
text or paper.

\section{Plagiarism}

The class includes a section about plagiarism which you must adhere
to. Copying text without proper citation is considered cheating and we
will assign the grade ``F'' for the paper if we find you do it. It is
in your responsibility to make sure plagiarism does not occur. Please
be aware that our checks are better than the once provided by turnitin
or other online checkers. Excuses such as ``I did not have time'' or
``I forgot'' can not apply as you have enough time to prepare the
paper and must not forget. 

\section{Check}

make sure just as in previous assignments that you check your paper
with chktex and lacheck. Fix the errors that you see. Some of the
errors may be ok, but in general make sure you address all of them. If
in doubt work with the TA. Simply use

\begin{verbatim}
make check
\end{verbatim}

We include in the handbook a list with common issues that we see when
students submit papers. One particular important issue is not to use
the underscore in bibtex labels. It is your responsibility to check
the paper for the issues indicated.

To check bibliographies simply use

\begin{verbatim}
pdflatex report
bibtex report
\end{verbatim}

You will see the errors and warning son the screen address them

TA's will in addition use a special test checking for additional
format issues such as detecting if you used labels and refs for
floats. You are welcome to also try this test, but we provide it
without explanation as no explanation is needed since if you followed
the instructions on floats there should be no issues. If you like to
do that test, you can use  

\begin{verbatim}
make check-ta
\end{verbatim}

\section{Convenient Setup}

If you do not have already a paper dir in your repository, here is a
way to create one. Replace the hid-sp18-000 with your hid.

\begin{verbatim}
export HID=hid-sp18-000
mkdir -p ~/github/cloudmesh-community
cd ~/github/cloudmesh-community
git clone https://github.com/cloudmesh-community/hid-sample.git
git clone https://github.com/cloudmesh-community/$HID.git
\end{verbatim}

Next copy the paper example

\begin{verbatim}
cp -r hid-sample/paper $HID
cd $HID
git add paper
git commit -m "add the paper directory" paper
git push
\end{verbatim}

Make sure there is no \verb|/| behind the paper in the cp command or you mess up the
copy process.


\section{Creating the PDF}

The PDF\ can be created simply with 

\begin{verbatim}
make clean
\end{verbatim}



{\bf UNDER NO CIRCUMSTANCES ARE YOU ALLOWED TO CHECK IN YOUR PDF OR
  TEMPORARY LATEX FILES INTO GITHUB. GIT NEEDS TO STAY CLEAN AND ONLY
  CONTAIN THE SOURCES.}

We will deduct points if you do violate this.

\section{Conclusion}

Put here an conclusion. Conclusion and abstracts must not have any
citations in the section.


\begin{acks}

  The authors would like to thank Dr.~Gregor~von~Laszewski for his
  support and suggestions to write this paper.

\end{acks}

\bibliographystyle{ACM-Reference-Format}
\bibliography{report} 

